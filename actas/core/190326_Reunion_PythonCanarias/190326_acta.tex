\documentclass[a4paper, 12pt]{article}

\usepackage[utf8]{inputenc}
\usepackage[spanish]{babel}
\usepackage{float}
\usepackage{hyperref}
\usepackage{tocloft}
\usepackage{titling}
\usepackage{eurosym}
\usepackage{bookmark}

\renewcommand{\labelitemi}{$\bullet$}
\renewcommand{\labelitemii}{$\circ$}
\renewcommand{\labelitemiii}{--}
\renewcommand{\cftsecleader}{\cftdotfill{\cftdotsep}}
\newcommand{\specialcell}[2][c]{%
    \begin{tabular}[#1]{@{}c@{}}#2\end{tabular}
}

\setlength{\droptitle}{-10em}

\hypersetup{
    colorlinks=true,
    linkcolor=black,
    filecolor=magenta,      
    urlcolor=cyan,
}

\hyphenation{Python-Canarias}
\hyphenation{Py-Day}
\hyphenation{Py-thon}

\title{\huge \textbf{Acta de reunión} \\ \textit{Python Canarias}}
\date{\textbf{26 de marzo de 2019}}
\author{
    Alejandro Moreno Alberto \and
    Israel Santana Alemán \and
    Jaime Iván Juanes Prieto \and
    Luis Cabrera Sauco \and
    Sergio Delgado Quintero
}

\begin{document}

\renewcommand{\contentsname}{Orden del día}

\maketitle

En \textit{Las Palmas de Gran Canaria}, siendo las \textit{18:00h} de la fecha arriba indicada se reúnen en \textit{Edosoft} los miembros de Python Canarias arriba indicados, a fin de tratar el siguiente orden del día:

\tableofcontents

\section{Lectura y aprobación, si procede, del acta anterior}

Se aprueba el acta anterior.

\section{Pylikes}

\begin{itemize}
    \item Se decide hacer el \textit{Pylikes} (pronunciado ``pailiques'') el día \textbf{26 de abril de 2019}.
    \item El lugar de celebración propuesto es \textit{Secret-Source} \url{https://www.secret-source.eu/}, un coworking situado cerca del Cabildo en Las Palmas de Gran Canaria.
    \item Se ha creado un tablero de \textit{Trello} con las tareas del evento: \url{https://trello.com/b/zXkHePL5/pyliques-lpa-2019}.
    \item La idea es impartir un par (no necesariamente 2) de charlas rápidas.
    \item Se tratará de buscar ponentes a través del grupo de \textit{Telegram} y de la cuenta de \textit{Twitter} de Python Canarias.
    \item Se decide cobrar entrada de 4\euro.
\end{itemize}

\section{Micropython}

\begin{itemize}
    \item Se decide hacer el taller de \textit{Micropython} el día \textbf{25 de mayo de 2019} por la mañana.
    \item Queda pendiente por decidir el lugar de celebración.
    \item Se reutilizará el repo de $\mu$Python desarrollado por Python Canarias \url{https://github.com/pythoncanarias/upython}, actualizándolo en lo necesario.
    \item El taller se impartirá por \textit{Iván Juanes} e \textit{Israel Santana}. En caso de ausencia queda como suplemente \textit{Alejandro Moreno}.
    \item Israel Santana ya dispone de \textit{10 placas ESP8266} que se podrían utilizar en el taller.
\end{itemize}

\section{PyDay}

\begin{itemize}
    \item Hay intención de realizar este año el PyDay en Gran Canaria.
    \item La fecha propuesta es el \textbf{16 de noviembre de 2019}.
    \item Se plantean lugares como la \href{https://www.spegc.org/}{SPEGC} o la \href{https://www.ulpgc.es}{ULPGC}.
    \item Hay un contacto interesante en la ULPGC. Se trata de un profesor \textit{Alexis Quesada} que podría ser de gran ayuda para la celebración allí del evento.
\end{itemize}

\section{Cuestiones relativas a la asociación}

\begin{itemize}
    \item Para estos eventos ``minoritarios'' se plantea una única modalidad de patrocinio de 50\euro\ cuya contrapartida es la visibilización de logotipo y marca de la organización en la web del evento.
    \item Se decide invitar al grupo del \textit{CORE} a \textit{Frank Sosa} al considerarlo una persona que puede ayudar en la gestión de los eventos de la asociación.
    \item A mediados de Abril, nuestro presidente \textit{Juan Ignacio Rodríguez} regresará a Tenerife (``cual hijo pródigo''). Aprovechando su llegada se intentará implementar el módulo en la web de \textit{alta de socios}. En cualquier caso también se plantea que, para los posibles descuentos en eventos cercanos, se podría realizar la gestión de socios mediante un \textit{PDF editable} hasta que se termine el desarrollo del módulo correspondiente de la web.
\end{itemize}



\section{Ruegos y preguntas}

No hay ruegos ni preguntas.

% ================================================================================================

\vspace{1cm}
\hrule
\vspace{3mm}

Una vez tratados todos los puntos, se levanta la sesión cuando son las \textit{19:30h} en lugar y fecha arriba indicados.

\begin{flushright}
El secretario

Sergio Delgado Quintero
\end{flushright}

\end{document}
