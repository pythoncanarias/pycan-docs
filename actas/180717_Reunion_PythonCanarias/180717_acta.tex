\documentclass[a4paper, 12pt]{article}

\usepackage[utf8]{inputenc}
\usepackage[spanish]{babel}
\usepackage{float}
\usepackage{hyperref}
\usepackage{tocloft}
\usepackage{titling}
\usepackage{eurosym}

\renewcommand{\labelitemi}{$\bullet$}
\renewcommand{\labelitemii}{$\circ$}
\renewcommand{\labelitemiii}{--}
\renewcommand{\cftsecleader}{\cftdotfill{\cftdotsep}}

\setlength{\droptitle}{-10em}

\hypersetup{
    colorlinks=true,
    linkcolor=black,
    filecolor=magenta,      
    urlcolor=cyan,
}

\hyphenation{Python-Canarias}
\hyphenation{Py-Day}

\title{\huge \textbf{Acta de reunión} \\ \textit{Python Canarias}}
\date{\textbf{18 de julio de 2018}}
\author{
    Alejandro Samarín Pérez \and
    Juan Ignacio Rodríguez de León\ \and 
    Héctor Manuel Figueras Hernández \and
    Lucas Grillo Lorenzo \and
    Raúl Marrero Rodríguez \and
    Sara Báez García \and
    Sergio Delgado Quintero
}

\begin{document}

\renewcommand{\contentsname}{Orden del día}

\maketitle

En \textit{San Cristóbal de La Laguna}, siendo las \textit{17:30h} de la fecha arriba indicada se reúnen en \textit{Kreitek} los miembros de Python Canarias arriba indicados, a fin de tratar el siguiente orden del día:

\tableofcontents

\section{Lectura y aprobación, si procede, del acta anterior}

Se aprueba el acta anterior.

\section{Cuestiones organizativas sobre PyDay 2018}

\subsection*{Venta de entradas}

\begin{itemize}
    \item Se decide por unanimidad vender las entradas a través de nuestra propia plataforma. Juan Ignacio ya ha desarrollado una primera versión utilizando \textit{Django} y la pasarela de pago \textit{Stripe}.
    \item Los motivos que nos llevan a esta decisión son los siguientes:
    \begin{itemize}
        \item Vender distintos tipos de entradas.
        \item Stripe tiene las comisiones más bajas de todos los sistemas de venta de entrada que hemos analizado.
        \item Flexibilidad de captura de datos y gestión de los mismos.
        \item Aprovechar el sistema para cobrar las cuotas de los socios.
        \item El dinero lo tendríamos, como máximo, de 7 a 10 días después del pago, mientras que en la mayoría de los sistemas de venta de entradas es al finalizar el evento.
    \end{itemize}
    \item Se plantea una idea de implementar en la web una sección donde los sponsors puedan pagar a través de Stripe. Se añade la correspondiente \textit{issue} en el GitHub de la asociación.
    \item Se decide reservar \textit{20 entradas para los patrocinadores}, que si al final no se usan, se pueden vender.
    \item Se acuerda sacar \textit{24 entradas early-access} a 10\euro. Estas entradas tienen un precio menor porque todavía no está publicada toda la información en la web, y los compradores van un poco "a ciegas".
\end{itemize}

\subsection*{Call for Papers}

\begin{itemize}
    \item Tenemos 4 propuestas de personas extranjeras, probablemente indios:
    \begin{itemize}
        \item \href{https://twitter.com/harshulrobo}{Harshul Jain}
        \item \href{https://twitter.com/m_m_murad}{Mohammad Murad}
        \item \href{https://www.linkedin.com/in/rahul-arulkumaran}{Rahul Arulkumaran}
        \item \href{https://www.linkedin.com/in/alizishaan-khatri-32a20637}{Alizishaan Khatri}
        \item \href{https://www.linkedin.com/in/aabir}{Aabir Abubaker Kar}
    \end{itemize}
    \item A día de hoy, después de un correo de bienvenida que hemos enviado, sólo hemos recibido respuesta de \textit{Harshul Jain} y de \textit{Alizishaan Khatri}. Con Harshul Jain no hay ningún problema de desplazamiento, pero Alizishaan Khatri ha solicitado una carta de invitación para poder gestionar su visado.
    \item Queda pendiente hablar con estas dos últimas personas para ver desde qué país se desplazarían y qué pasaporte tienen.
    \item Se acuerda tener, al menos, \textit{2 charlas de backup}, por si acaso se nos cae algún ponente de última hora. Juan Ignacio se propone con una ponencia de REDIS. Queda pendiente una segunda charla.
    \item Sergio ha contactado con dos posibles ponentes de península: \href{https://twitter.com/SoyGema}{Gema Parreño} y \href{https://twitter.com/aljesusg}{Alberto Gutiérrez}. Ellos han confirmado su presencia pero necesitamos financiación para poder cubrir su desplazamiento.
    \item Juan Ignacio contactará con el \textit{GDG de Canarias} para ver la posibilidad de que ellos puedan cubrir los gastos de estos dos ponentes a cambio de un patrocinio.
    \item Una vez que finalice el CFP (31 de julio de 2018) se acuerda hacer un mailing a los ponentes seleccionados explicando que \textit{las charlas deben tener 40 minutos más 10 minutos de preguntas}. Así se podrán preparar una ponencia acorde a la duración establecida.
\end{itemize}

\subsection*{Difusión}

\begin{itemize}
    \item El \textit{mailing} que estamos haciendo para captar patrocinadores no está dando el fruto esperado. Se acuerda buscar contactos más directos en empresas para que podamos tener más posibilidades de conseguir financiación.
    \item Se acuerda ir \textit{poniendo en la web del evento los patrocinadores} que ya tenemos confirmados. Para ello se solicitará a los patrocinadores el logotipo de la empresa en formato vectorial o PNG de más de 500px en su defecto.
    \item Juan Ignacio se ofrece a hacer el \textit{cartel del evento}, pero se acuerda igualmente pedir presupuesto a una diseñadora gráfica para ver si podemos afrontar el pago.
    \item \textit{Carlos Blé} se ha ofrecido a colaborar puntualmente con la asociación. De este modo, se acuerda proponerle ayuda en la búsqueda de patrocinadores, de manera que él contacte primero con empresas y luego nos deje el contacto para cerrar los detalles.
    \item Héctor Álvarez asistirá a la EuroPython. Allí asistirá a una comida donde varias asociaciones de Python en Europa estarán juntas para compartir conocimientos. De cara a la difusión se acuerda adaptar \href{https://slides.com/sdelquin/pythoncanarias}{la presentación de la asociación} que se hizo en el PyDay Gran Canaria 2017.
    \item Se acuerda hacer mailing a los patrocinadores pidiendo información del material publicitario que van a incluir en el \textit{welcome-pack}. Así podemos ir gestionando el contenido del mismo.
    \item Se podría intentar conseguir \textit{botellas de agua} para incluir en los welcome-pack. Esto nos lleva a contactar con otras empresas: Fuente Alta, Libbys, Embotelladora de Canarias, Arehucas, Coca-Cola, Tirma.
    \item Queda pendiente hacer el \textit{flyer} del evento, que podría ser la agenda junto con los patrocinadores.
\end{itemize}

\subsection*{Página web}

\begin{itemize}
    \item Hay que incluir la \textit{política de privacidad} y las \textit{condiciones legales} en la web.
    \item Además hay que añadir \textit{casillas de verificación} al hacer la compra de una entrada, en relación a protección de datos.
    \item Se decide homogeneizar la web de Python Canarias y gestionarlo todo desde la aplicación \textit{Django} que ya se está elaborando para la venta de entradas. Así, tendremos URLs del estilo:
    \begin{itemize}
        \item Eventos de la asociación:
        \begin{itemize}
            \item \texttt{pythoncanarias.es/events/pydaytf2018/}
            \item \texttt{pythoncanarias.es/events/pydaytf2018/tickets/}
            \item \ldots
        \end{itemize}
        \item Gestión de socios: \texttt{pythoncanarias.es/members/}
        \item Ofertas de trabajo: \texttt{pythoncanarias.es/jobs/}
        \item \ldots
    \end{itemize}
\end{itemize}

\section{Cuestiones relativas a la asociación Python Canarias}

\begin{itemize}
    \item Juan Ignacio comunica que en Agosto empezará a trabajar para \textit{OctopusLab} en Londres. Aún así, él seguirá siendo el presidente de la asociación y seguirá colaborando activamente.
    \item Dada la marcha de Juan Ignacio a Londres, se hace necesario buscar una nueva dirección fiscal que aportar a la asociación. Quedamos a la espera de la resolución del expediente en el registro de asociaciones canarias.
    \item Después de haber analizado la situación actual, se decide no solicitar la celebración de la PyConES 2019 en Tenerife.
\end{itemize}

\section{Ruegos y preguntas}

No hay ruegos ni preguntas.

% ================================================================================================

\vspace{1cm}
\hrule
\vspace{3mm}

Una vez tratados todos los puntos, se levanta la sesión cuando son las \textit{20:20h} en lugar y fecha arriba indicados.

\begin{flushright}
El secretario

Sergio Delgado Quintero
\end{flushright}

\end{document}
