\documentclass[a4paper, 12pt]{article}

\usepackage[utf8]{inputenc}
\usepackage[spanish]{babel}
\usepackage{float}
\usepackage{hyperref}
\usepackage{tocloft}
\usepackage{titling}
\usepackage{eurosym}

\usepackage{todonotes} % Verificar si está disponible.

\renewcommand{\labelitemi}{$\bullet$}
\renewcommand{\labelitemii}{$\circ$}
\renewcommand{\labelitemiii}{--}
\renewcommand{\cftsecleader}{\cftdotfill{\cftdotsep}}

\setlength{\droptitle}{-10em}

\hypersetup{
    colorlinks=true,
    linkcolor=black,
    filecolor=magenta,      
    urlcolor=cyan,
}

\hyphenation{Python-Canarias}
% \hyphenation{Micro-Python}
%\hyphenation{Py-Day}

\title{\huge \textbf{Acta de reunión} \\ \textit{Python Canarias}}
\date{\textbf{1 de Junio de 2018}}
\author{
    Israel (@kamaxeon)\\
    Iván Juanes (@Kerby)\\ 
    Luis Cabrera (@lcabrera)
}

\begin{document}

\renewcommand{\contentsname}{Orden del día}

\maketitle

En \textit{Las Palmas de Gran Canaria}, siendo las \textit{21:00h} de la fecha arriba indicada se reúnen en el \textit{Centro Comercial 7 Palmas} los miembros de Python Canarias arriba indicados, a fin de tratar el siguiente orden del día:

\tableofcontents

% \clearpage

\section{Lectura y aprobación, si procede, del acta anterior}

Al ser la primera acta que levantamos, no hizo falta aprobar actas anteriores.

\section{Cuestiones organizativas sobre el PyBirras 2018}

\begin{itemize}

	\item Se fija el precio de inscripción en 2\euro.
	\item Tenemos 4 o 5 ponentes\footnote{Concretar nombres y confirmaciones}, que a 20 minutos por monólogo\footnote{Este término salió durante el intercambio de opiniones sobre cual debería ser el enfoque adecuado para un PyBirras} da para hora y media - 2 horas 
	\item Al final del evento evaluaremos qué tal es \textit{hacerlo en un local de formación en vez de una cervecería}\footnote{Relativo a cómo marcar la diferencia entre los PyBirras y otros tipos de eventos.}, y ver posibles propuestas de mejora.
	\item Hacer el siguiente PyBirras lo antes posible, uno o dos meses
	\item Reflexionar sobre una orientación menos formal del evento\footnote{En línea con la idea de marcar la diferencia entre los PyBirras y otros tipo de eventos}
	\item Para comenzar el primer PyBirras con buen pié, se acordó que las cervezas se entregaran al comienzo del evento.

\end{itemize}

Tambien se habló de:
\begin{itemize}
	\item Difusión:
	\begin{itemize}
		\item Contactos con la ULPGC
		\item Contactar con el C.O.I.T.I.C.
		\item Posible difusión en Centros de Formación Profesional\footnote{¿Iván?}
		\item Diferentes canales \textbf{Meetup} y similares
		\item Canales RRSS de Python Canarias
	\end{itemize}
	\item Página web y formulario. 
	\item Rajoy
\end{itemize}



\section{Ruegos y preguntas}

Nada especial.

% ================================================================================================

%\vspace{1cm}
%\hrule
%\vspace{3mm}
%
%Una vez tratados todos los puntos, se levanta la sesión cuando son las \textit{19:30h} en lugar y fecha arriba indicados.
%
%\begin{flushright}
%El secretario
%
%Sergio Delgado Quintero
%\end{flushright}

\end{document}
