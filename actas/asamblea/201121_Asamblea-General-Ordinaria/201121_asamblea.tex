\documentclass[a4paper, 12pt]{article}

\usepackage[T1]{fontenc}
\usepackage[utf8]{inputenc}
\usepackage[spanish]{babel}
\usepackage{float}
\usepackage{hyperref}
\usepackage{tocloft}
\usepackage{titling}
\usepackage{eurosym}
\usepackage{bookmark}

\renewcommand{\labelitemi}{$\bullet$}
\renewcommand{\labelitemii}{$\circ$}
\renewcommand{\labelitemiii}{--}
\renewcommand{\cftsecleader}{\cftdotfill{\cftdotsep}}
\newcommand{\specialcell}[2][c]{%
    \begin{tabular}[#1]{@{}c@{}}#2\end{tabular}
}

\setlength{\droptitle}{-10em}

\hypersetup{
    colorlinks=true,
    linkcolor=black,
    filecolor=magenta,      
    urlcolor=cyan,
}

\hyphenation{Python-Canarias}
\hyphenation{Py-Day}
\hyphenation{Py-thon}

\title{\huge \textbf{Acta de reunión} \\ Asamblea General Ordinaria \\ \textit{Python Canarias}}
\date{\textbf{21 de noviembre de 2020}}
\author{---}

\begin{document}

\renewcommand{\contentsname}{Orden del día}

\maketitle

\begin{enumerate}
    \item Cayetano Hernández Osma.
    \item Héctor Álvarez Alonso.
    \item Israel Santana Alemán.
    \item Juan Ignacio Rodríguez de León.
    \item Raúl Anatol Morales Martín.
    \item Raúl Marrero Rodríguez.
    \item Sara Báez García.
    \item Sergio Delgado Quintero.
    \item Wendolín Damián González.
    \item Yodra López Herrera.
\end{enumerate}

Vía \textit{Google Meet}, siendo las \textit{10:10h (Atlantic/Canary)} de la fecha arriba indicada se reúnen los socios y socias de Python Canarias arriba indicados, a fin de tratar el siguiente orden del día:

\tableofcontents

\section{Aprobación, si procede, de las cuentas económicas del ejercicio 2019}

El tesorero, Héctor Álvarez, procede a explicar los balances, memoria y cuentas de pérdidas y ganancias de nuestro primer ejercicio económico del año 2019. Tras la presentación y explicación de las cuentas económicas del ejercicio 2019, sin que haya preguntas al respecto, se pasa a aprobar las cuentas del ejercicio 2019 por unanimidad de la asamblea, sin que ninguno se pronuncie en contra.

\section{Aprobación, si procede, de la memoria anual de actividades del año 2019}

El secretario, Sergio Delgado, presenta las actividades desarrolladas durante el pasado año 2019. Fueron cuatro:

\begin{enumerate}
    \item Taller de Flask para Mujeres (Abril de 2019).
    \item Taller de Micropython (Mayo de 2019).
    \item PyBirras (Julio de 2019).
    \item PyDay Gran Canaria (Noviembre de 2019).
\end{enumerate}

Se aprueba la memoria de actividades por unanimidad de la asamblea.

\section{Aprobación, si procede, del presupuesto para el año 2020}

El tesorero, Héctor Álvarez, presenta el presupuesto para el año 2020. Pide disculpas por hacerlo tan tarde, pero se ha debido a la situación extraordinaria de pandemia. Se aprueba el presupuesto por unanimidad de la asamblea.

\section{Aprobación, si procede, del plan general de actuación para el año 2020}

El secretario, Sergio Delgado, presenta el plan general de actuación para este presente año 2020. La única actividad desarrollada ha sido ``PyBirras Batallitas'', un evento online celebrado en Junio de 2020.

Pide disculpas por presentar este plan tan tarde, pero se ha debido a la situación extraordinaria de pandemia.

Se aprueba el plan general de actuación por unanimidad de la asamblea.

\section{Modificación del artículo 2 de los estatutos}

Se explica a la asamblea la propuesta de modificar el artículo 2 de los estatutos, referente al domicilio social, para hacerlo coincidir con la dirección de la asesoría que tenemos actualmente.

Se aprueba, por unanimidad de la asamblea, la modificación del artículo 2 de los estatutos con la siguiente redacción:\\

\hrule
\vspace{3mm}

La asociación tendrá su domicilio social en la siguiente dirección postal:

\begin{quotation}
    Calle Los Molinos, 1 - Bajo derecha

    38005

    Santa Cruz de Tenerife

    España
\end{quotation}
\hrule

\section{Modificación del artículo 13 de los estatutos}

Se explica a la asamblea la propuesta de modificar el artículo 13 de los estatutos, referente a la publicidad de los datos, para permitir, en un futuro, que se pueda verificar la condición de socio/a en la web.

Se aprueba, por unanimidad de la asamblea, la modificación del artículo 13 de los estatutos con la siguiente redacción:\\

\hrule
\vspace{3mm}

Se podrá verificar la condición de socio/a en la web de Python Canarias, según los mecanismos que se consideren oportunos de acuerdo a la legislación vigente.

\vspace{3mm}
\hrule

\section{Modificación del artículo 18.1 de los estatutos}

Se explica a la asamblea la propuesta de modificar el artículo 18.1 de los estatutos, referente a la estructura del órgano de representación, para prescindir de la figura de vocal en el mismo, ya que no se considera necesario.

Se aprueba, por unanimidad de la asamblea, la modificación del artículo 18.1 de los estatutos con la siguiente redacción:\\

\hrule
\vspace{3mm}

El órgano de representación será la junta directiva y estará integrada por los siguientes miembros:

\begin{enumerate}
    \item Un/a presidente/a.  
    \item Un/a vicepresidente/a.  
    \item Un/a secretario/a.  
    \item Un/a tesorero/a.  
\end{enumerate}

\hrule

\section{Modificación del artículo 19.4 de los estatutos}

Se explica a la asamblea la propuesta de modificar el artículo 19.4 de los estatutos, referente a la convocatoria de la asamblea general, con el objetivo de dar algo más de tiempo para celebrarla al comienzo de cada año.

Se aprueba, por unanimidad de la asamblea, la modificación del artículo 19.4 de los estatutos con la siguiente redacción:\\

\hrule
\vspace{3mm}

La asamblea general deberá ser convocada al menos en sesión ordinaria una vez al año, dentro del primer cuatrimestre anual, incluyendo, al menos, el siguiente punto del orden del día: examinar y aprobar la liquidación anual de las cuentas del ejercicio anterior, el presupuesto del ejercicio corriente y la memoria de actividades del ejercicio anterior.

Asimismo, se podrá convocar en sesión extraordinaria cuando así lo acuerde el órgano de representación.

Las asambleas generales se celebrarán previa convocatoria del presidente, acompañada del orden del día consignando lugar, fecha y hora. En el caso de que la convocatoria no incluyese el lugar de celebración se entenderá a todos los efectos el domicilio social.

La toma de decisiones se realizará por mayoría de votos de los socios asistentes.

\vspace{3mm}
\hrule

\section{Modificación del artículo 24 de los estatutos}

Se explica a la asamblea la propuesta de modificar el artículo 24 de los estatutos, referente a la cuantía de las cuotas, para eliminar la cuota de socio/a conjunta de Python Canarias y Python España. Se ha consultado esta posibilidad a Python España y resulta complicado a nivel operativo/logístico mantener esta vinculación conjunta.

Se aprueba, por unanimidad de la asamblea, la modificación del artículo 24 de los estatutos con la siguiente redacción:\\

\hrule
\vspace{3mm}

Existirán los siguientes tipos de cuotas para los socios:

\begin{itemize}
    \item 20\euro\ al año: precio general.
    \item 10\euro\ al año: precio para estudiantes y/o desempleados.
\end{itemize}

La condición de estudiante y/o desempleado deberá ser justificada documentalmente.

\vspace{3mm}
\hrule

\section{Modificación del artículo 20 de los estatutos}

Se explica a la asamblea la propuesta de modificar el artículo 20 de los estatutos, referente al procedimiento de modificación de los estatutos, para tratar de afianzar los estatutos y buscar mayor consenso en su alteración.

Se aprueba, por unanimidad de la asamblea, la modificación del artículo 20 de los estatutos con la siguiente redacción:\\

\hrule
\vspace{3mm}

Los estatutos de la asociación podrán ser modificados cuando resulte conveniente a los intereses de la misma, por tres cuartas partes de los votos asistentes en la asamblea general convocada al efecto.

\vspace{3mm}
\hrule

\section{Renovación de cargos en la junta directiva}

El tesorero, Héctor Álvarez, manifiesta su voluntad de dejar su cargo en la junta directiva por motivos personales.

La socia Sara Báez García se propone para el cargo de tesorera. Tras un breve debate al respecto, se llega al siguiente acuerdo unánime de la asamblea con respecto a la renovación de cargos:

\begin{itemize}
    \item \textit{Presidente}: Juan Ignacio Rodríguez de León.
    \item \textit{Vicepresidente}: Israel Santana Alemán.
    \item \textit{Tesorera}: \textbf{Sara Báez García}.
    \item \textit{Secretario}: Sergio Delgado Quintero.
\end{itemize}

\section{Ratificación de altas y bajas de socios y socias}

El secretario, Sergio Delgado, presenta a la asociación las altas y las bajas de socios y socias desde el año 2018, con el siguiente balance:

\begin{itemize}
    \item 2018: 8 altas y 0 bajas.
    \item 2019: 10 altas y 4 bajas.
    \item 2020: 7 altas y 0 bajas.
\end{itemize}

A día de hoy, la asociación cuenta con 21 socios y socias.

\section{Otra información de interés}

Se tratan algunas cuestiones referentes a la asociación:

\begin{itemize}
    \item \textbf{Página web}: Se solicita ayuda a socios y socias para colaborar en el mantenimiento de la página web. Se hace un agradecimiento al socio Jacobo de Vera, que ha ayudado mucho en ``dockerizar'' el proyecto y facilitar la entrada de nuevas personas al desarrollo.
    \item \textbf{Subvenciones}: Se solicita colaboración a socios y socias para concurrir a posibles subvenciones públicas o aportar información al respecto.
    \item \textbf{Espacio físico}: Se solicita colaboración a socios y socias para ver si podemos encontrar algún local para la asociación. Tratar de contactar con ayuntamientos a tal fin.
    \item \textbf{Evento ciencia de datos}: El secretario, Sergio Delgado, informa de un evento online sobre ciencia de datos que se desarrollará a finales de enero. Esta ``conferencia'' se hará en colaboración con R Canarias, Python España, Python Alicante, Python Castilla y León, PyData Salamanca, Python Barcelona y Python Málaga.
    \item \textbf{Próximas acciones}: Algunos socios y algunas socias proponen posibles acciones de cara al próximo año:
    \begin{itemize}
        \item Microtaller en desarrollo web, que pueda servir como punto de entrada para ayudar en la web de la asociación \href{https://pythoncanarias.es}{pythoncanarias.es}.
        \item Coding Dojo con pequeñas catas y ejercicios para aprender.
        \item Microeventos más enfocados a experiencias que a conocimientos.
    \end{itemize}
\end{itemize}

\section{Sugerencias y preguntas}

No hay sugerencias ni preguntas.

% ================================================================================================

\vspace{1cm}
\hrule
\vspace{3mm}

Una vez tratados todos los puntos, se levanta la sesión cuando son las \textit{11:53h} en lugar y fecha arriba indicados.

\vspace{1cm}

\begin{table}[h]
    \begin{tabular}{p{9cm}p{9cm}}
        VºBº Presidencia & El secretario \\
        \vspace{3cm} & \vspace{3cm} \\
        Juan Ignacio Rodríguez de León & Sergio Delgado Quintero \\
    \end{tabular}
\end{table}

\end{document}
