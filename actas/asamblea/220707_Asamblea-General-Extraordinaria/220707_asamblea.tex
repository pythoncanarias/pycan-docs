\documentclass[a4paper,12pt]{article}

\usepackage[T1]{fontenc}
\usepackage[utf8]{inputenc}
\usepackage[spanish]{babel}
\usepackage{float}
\usepackage{hyperref}
\usepackage{tocloft}
\usepackage{titling}
\usepackage{eurosym}
\usepackage{bookmark}

\renewcommand{\labelitemi}{$\bullet$}
\renewcommand{\labelitemii}{$\circ$}
\renewcommand{\labelitemiii}{--}
\renewcommand{\cftsecleader}{\cftdotfill{\cftdotsep}}
\newcommand{\specialcell}[2][c]{%
    \begin{tabular}[#1]{@{}c@{}}#2\end{tabular}
}

\setlength{\droptitle}{-10em}

\hypersetup{
    colorlinks=true,
    linkcolor=black,
    filecolor=magenta,      
    urlcolor=cyan,
}

\hyphenation{Python-Canarias}
\hyphenation{Py-Day}
\hyphenation{Py-thon}

\title{\huge \textbf{Acta de reunión} \\ Asamblea General Extraordinaria \\ \textit{Python Canarias}}
\date{\textbf{7 de julio de 2022}}
\author{---}

\begin{document}

\renewcommand{\contentsname}{Orden del día}

\maketitle

\begin{enumerate}
    \item Héctor Álvarez Alonso.
    \item Israel Santana Alemán.
    \item Jesús Miguel Torres Jorge.
    \item Luis Leoncio Cabrera Sauco.
    \item Raúl Marrero Rodríguez.
    \item Sara Báez García.
    \item Sergio Delgado Quintero.
    \item Wendolín Damián González.
\end{enumerate}

Vía \textit{Google Meet}, siendo las \textit{19:00h (Atlantic/Canary)} de la fecha arriba indicada se reúnen los socios y socias de Python Canarias arriba indicados, a fin de tratar el siguiente orden del día:

\tableofcontents

\section{Candidatura a PyConES 2023}

Sergio Delgado explica los puntos principales de la candidatura de Canarias a la conferencia nacional de Python en el año 2023.\\

Las fechas propuestas para la celebración de la \textit{PyConES23} son 6, 7 y 8 de octubre de 2023.\\

En cuanto a la sede. Después de haber pedido presupuestos a muchos centros de congresos de Tenerife sin obtener respuesta, se realiza la gestión con la Universidad de La Laguna. En este sentido todo han sido facilidades. El pasado 20 de junio se celebró una reunión a la que asistieron Rosa Aguilar (rectora de la ULL), Jordi Riera (Vicerrector de Agenda Digital, Modernización y Campus Central), Juan Ignacio Rodríguez (presidente de Python Canarias) y Sergio Delgado (secretario de Python Canarias). En dicha reunión se cerró el acuerdo para disponer del Aulario General de la ULL en el campus de Guajara.\\

Teniendo en cuentas datos de venta de entradas de la próxima PyConES Granada 2022, se estima en 800 personas la afluencia de participantes a la conferencia que se celebre en Canarias.\\

El pasado día 29 de junio, tanto Juan Ignacio como Sergio visitaron las instalaciones del Aulario de Guajara acompañados de David Ojeda (conserje de las instalaciones). Allí se pudo comprobar la adecuación de los espacios a los requerimientos del evento. Por ello, se llevó a cabo la reserva inicial en las fechas arriba indicadas de las aulas más relevantes:

\begin{itemize}
    \item Aula Magna (600 personas)
    \item Aula 0.4 (190 personas).
    \item Aula 0.5 (190 personas).
    \item Aula 1.9 (140 personas).
    \item Aula 1.10 (140 personas).
\end{itemize}

Sergio Delgado continua comentando que se ha solicitado ya presupuestos a distintas empresas en relación con catering, audiovisuales o diseño.\\

En el tema catering hay que tener presente la logística para dar de comer a un volumen tan grande de gente. Un aspecto importante a tener en cuenta es la posible contratación de catering con la misma empresa que gestiona la propia cafetería del Aulario, ya que esto permitiría utilizar ese espacio, que de otra forma, no sería posible.\\

El plazo de presentación de la candidatura es el 1 de septiembre de 2022. Ya se está empezando a redactar el documento que incluye aspectos como:

\begin{itemize}
    \item Equipo organizador.
    \item Fechas.
    \item Presupuesto.
    \item Sede.
    \item Iniciativas para promover la diversidad.
    \item Gestión económica.
    \item Información sobre la localidad.
    \item Filosofía del evento.
\end{itemize}

Sergio termina la exposición indicando una serie de items a tener en cuenta en la propuesta de sede: Entradas / Registro, Welcome pack, Keynoters, Apertura, Diversidad, Catering, Alojamiento / Transporte, Patrocinios / Colaboraciones, Audiovisuales, Diseño / Cartelería, Regalos, Clausura y Web.\\

A partir de este punto, los socios y socias asistentes a la reunión realizan las siguientes aportaciones:

\begin{itemize}
    \item Posibilidad de llevar a cabo excursiones a centros de interés en Tenerife para pontentes y/o asistentes.
    \item Resulta interesante la propuesta de ``streaming online'' de pago.
    \item Se pregunta por el presupuesto que se maneja en este tipo de eventos y si los ingresos serán suficientes para cubrir los posibles gastos. En principio se considera que los ingresos por entradas y patrocinios serán suficientes para afrontar el evento.
    \item Tener en cuenta que si hay una institución pública detrás habría obligación de disponer de ILSE para personas sordas.
    \item Resulta conveniente recoger información específica de cada asistentes a la hora de la compra de la entrada: discapacidad, restricciones alimentarias, patologías, conciliación, etc. De cara a una mejor organización de los servicios.
    \item Se sugiere el contacto con \href{https://fasican.org/accesibilidad-en-medios-audiovisuales/}{FASICAN} para contratación de ILSE.
    \item También se propone contactar con la \href{https://lafast.org/}{FAST} para disponer de plátanos.
    \item Se pregunta por la accesibilidad de los espacios del Aulario de Guajara. En principio no hay ningún inconveniente en ese sentido.
    \item Se sugiere la lectura de las memorias de otras PyConES así como el aprendizaje de eventos tecnológicos que se han desarrollado aquí para mejorar y no partir de cero.
    \item Se propone ``cajita pycnic'' como nombre para la posible caja de almuerzo que se dé a asistentes.
\end{itemize}



% ================================================================================================

\vspace{1cm}
\hrule
\vspace{3mm}

Una vez tratados todos los puntos, se levanta la sesión cuando son las \textit{20:00h} en lugar y fecha arriba indicados.

\vspace{1cm}

\begin{table}[h]
    \begin{tabular}{p{9cm}p{9cm}}
        VºBº Presidencia & El secretario \\
        \vspace{3cm} & \vspace{3cm} \\
        Juan Ignacio Rodríguez de León & Sergio Delgado Quintero \\
    \end{tabular}
\end{table}

\end{document}
