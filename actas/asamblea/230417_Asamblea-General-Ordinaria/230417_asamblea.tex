\documentclass[a4paper,12pt]{article}

\usepackage[T1]{fontenc}
\usepackage[utf8]{inputenc}
\usepackage[spanish]{babel}
\usepackage{float}
\usepackage{hyperref}
\usepackage{tocloft}
\usepackage{titling}
\usepackage{eurosym}
\usepackage{bookmark}

\renewcommand{\labelitemi}{$\bullet$}
\renewcommand{\labelitemii}{$\circ$}
\renewcommand{\labelitemiii}{--}
\renewcommand{\cftsecleader}{\cftdotfill{\cftdotsep}}
\newcommand{\specialcell}[2][c]{%
    \begin{tabular}[#1]{@{}c@{}}#2\end{tabular}
}

\setlength{\droptitle}{-10em}

\hypersetup{
    colorlinks=true,
    linkcolor=black,
    filecolor=magenta,      
    urlcolor=cyan,
}

\hyphenation{Python-Canarias}
\hyphenation{Py-Day}
\hyphenation{Py-thon}

\title{\huge \textbf{Acta de reunión} \\ Asamblea General Ordinaria \\ \textit{Python Canarias}}
\date{\textbf{17 de abril de 2023}}
\author{---}

\begin{document}

\renewcommand{\contentsname}{Orden del día}

\maketitle

\begin{enumerate}
    \item Amanda María Del Olmo García.
    \item Eloy Pérez Reyes.
    \item Erick Massip Cano.
    \item Israel Santana Alemán.
    \item Jacobo de Vera Hernández.
    \item Juan Ignacio Rodríguez de León.
    \item Kevin Hierro Carrasco.
    \item Laeticia Dos Santos.
    \item Luis Leoncio Cabrera Sauco.
    \item Miguel Sánchez Ramos.
    \item Raúl Marrero Rodríguez.
    \item Sara Báez García.
    \item Sergio Delgado Quintero.
    \item Wendolín Damián González.
    \item Yeray Gutiérrez Cedrés.
\end{enumerate}

Vía \textit{Google Meet}, siendo las \textit{18:00h (Atlantic/Canary)} de la fecha arriba indicada se reúnen los socios y socias de Python Canarias previamente indicados, a fin de tratar el siguiente orden del día:

\tableofcontents

\vspace{1cm}

\section{Aprobación, si procede, de las cuentas económicas del ejercicio 2022}

Presentadas y explicadas las cuentas económicas del ejercicio 2022, resumidas en el siguiente cuadro:

\begin{center}
    \begin{tabular}{ | l | r | }
        \hline
        \textbf{Resumen de cuentas económicas} & \textbf{2022} \\ 
        \hline
        Total Ingresos & 436,09\euro \\  
        \hline
        Total Gastos & -334,61\euro \\  
        \hline
        \hline
        Diferencia & 101,48\euro \\  
        \hline
    \end{tabular}
\end{center}

Se aprueban las cuentas económicas por unanimidad de la asamblea.

\section{Aprobación, si procede, del presupuesto para el año 2023}

Presentado y explicado el presupuesto para el año 2023, resumido
en el siguiente cuadro:

\begin{center}
    \begin{tabular}{ | l | r | }
        \hline
        \textbf{Presupuesto} & \textbf{2023} \\ 
        \hline
        Total Ingresos & 600\euro \\  
        \hline
        Total Gastos & -294,71\euro \\  
        \hline
        \hline
        Diferencia & 305,29\euro \\  
        \hline
    \end{tabular}
\end{center}

Se aprueba el presupuesto por unanimidad de la asamblea.

\section{Aprobación, si procede, de la memoria anual de actividades del año 2022}

Se presentan las actividades desarrolladas durante el año 2022:

\begin{itemize}
    \item Se celebró un \href{https://medium.com/pythoncanarias/regreso-al-pybirras-6cb3513f0469}{Back to PyBirras} en modalidad presencial el pasado 9 de julio de 2022. Este evento tuvo tres charlas impartidas por Semidán Robaina Estévez, Esau Rodríguez y Juan Ignacio Rodríguez. Se celebró en el Museo de la Ciencia y el Cosmos con una afluencia de unas 30 personas.
    \item Se presentó la candidatura para la \href{https://2023.es.pycon.org}{PyConES23} el pasado 15 de agosto de 2022, recibiendo la confianza de Python España en la última conferencia nacional de Python celebrada en Granada en octubre de 2022.
\end{itemize}

Se aprueba la memoria de actividades por unanimidad de la asamblea.

\section{Aprobación, si procede, del plan general de actuación para el año 2023}

Se presentan las siguientes actividades:

\begin{itemize}
    \item \textbf{PyConES23}: Conferencia nacional de Python a celebrar en la Universidad de La Laguna (Tenerife) los días 6, 7 y 8 de octubre de 2023.
    \item \textbf{PyBirras Gran Canaria}: Se propone celebrar un PyBirras en Gran Canaria. El socio Luis Leoncio Cabrera Sauco se compromete a organizarlo.
    \item Contactar con \textbf{centros de formación profesional} para hacer difusión del lenguaje de programación Python entre profesorado y alumnado. El socio Kevin Hierro Carrasco se compromete a dirigirse al CIFP César Manrique en Tenerife.
\end{itemize}

Se aprueba el plan general de actuación por unanimidad de la asamblea.\\

\section{Renovación de cargos en la junta directiva}

Se traslada a la asamblea que en noviembre de 2022 se cumplieron 4 años de la junta directiva fundadora de la asociación. Según los estatutos éste es el período establecido para los cargos designados aunque pueden ser reelegidos. En cualquier caso se abre la puerta a la entrada de cualquier socio o socia que quiera ingresar en el órgano de representación.\\

No se produce ninguna renovación en la junta directiva.

\section{Ratificación de altas y bajas de socios y socias}

Desde la fundación de la asociación en noviembre de 2018 se han asociado 44 personas. A día de la redacción de esta acta se cuenta con 34 socios/as activos/as.\\

Relación de altas y bajas de socios y socias durante el año 2022:

\begin{center}
    \begin{tabular}{ | l | r | }
        \hline
        \textbf{Altas} & \textbf{Bajas} \\ 
        \hline
        \hline
        Ivelina Mirkova Blagoslavova & Rodolfo Illada \\  
        \hline
        Yeray Gutiérrez Cedrés & Alejandro Lorenzo Dávila \\  
        \hline
        Laeticia Dos Santos & \\  
        \hline
        Tanausú Hernández Yanes & \\  
        \hline
        Theofanis Petkos & \\  
        \hline
        John Kirwan & \\  
        \hline
        Silvia García Hernández & \\  
        \hline
        \hline
        8 & 2 \\  
        \hline
    \end{tabular}
\end{center}

Se ratifican las altas y bajas de socios y socias por unanimidad de la asamblea.

\section{Sugerencias y preguntas}

Distintos socios y socias sugieren algunas actividades a realizar por la asociación:

\begin{itemize}
    \item Preparar vídeos que permitan una formación inicial básica en Python de cara a dar respuesta a las necesidades de formación de determinado perfil de personas.
    \item Dinamizar la participación de las personas que constituyen el grupo de Telegram de Python Canarias, de cara a facilitarles su integración.
    \item Implementar un taller básico de iniciación a Python (presencial u online).
    \item Actividades tipo ``katas/coding-dojo'' para hacer online.
    \item Sinergia con la asociación AdaLoveDev de cara a desarrollar alguna actividad formativa presencial/online.
    \item Acercar el mundo del testing y en particular el TDD ``Test Driven Development'' a la comunidad de Python Canarias.
\end{itemize}

% ================================================================================================

\vspace{1cm}
\hrule
\vspace{3mm}

Una vez tratados todos los puntos, se levanta la sesión cuando son las \textit{19:45h} en lugar y fecha arriba indicados.

\vspace{1cm}

\begin{table}[h]
    \begin{tabular}{p{9cm}p{9cm}}
        VºBº Presidencia & El secretario \\
        \vspace{3cm} & \vspace{3cm} \\
        Juan Ignacio Rodríguez de León & Sergio Delgado Quintero \\
    \end{tabular}
\end{table}

\end{document}
