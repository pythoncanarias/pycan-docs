\documentclass[a4paper,12pt]{article}

\usepackage[T1]{fontenc}
\usepackage[utf8]{inputenc}
\usepackage[spanish]{babel}
\usepackage{float}
\usepackage{hyperref}
\usepackage{tocloft}
\usepackage{titling}
\usepackage{eurosym}
\usepackage{bookmark}

\renewcommand{\labelitemi}{$\bullet$}
\renewcommand{\labelitemii}{$\circ$}
\renewcommand{\labelitemiii}{--}
\renewcommand{\cftsecleader}{\cftdotfill{\cftdotsep}}
\newcommand{\specialcell}[2][c]{%
    \begin{tabular}[#1]{@{}c@{}}#2\end{tabular}
}

\setlength{\droptitle}{-10em}

\hypersetup{
    colorlinks=true,
    linkcolor=black,
    filecolor=magenta,      
    urlcolor=cyan,
}

\hyphenation{Python-Canarias}
\hyphenation{Py-Day}
\hyphenation{Py-thon}

\title{\huge \textbf{Acta de reunión} \\ Asamblea General Ordinaria \\ \textit{Python Canarias}}
\date{\textbf{29 de abril de 2021}}
\author{---}

\begin{document}

\renewcommand{\contentsname}{Orden del día}

\maketitle

\begin{enumerate}
    \item Héctor Álvarez Alonso.
    \item Israel Santana Alemán.
    \item José Lucas Grillo Lorenzo.
    \item Juan Ignacio Rodríguez de León.
    \item Luis Leoncio Cabrera Sauco.
    \item Raúl Marrero Rodríguez.
    \item Sara Báez García.
\end{enumerate}

Vía \textit{Google Meet}, siendo las \textit{18:00h (Atlantic/Canary)} de la fecha arriba indicada se reúnen los socios y socias de Python Canarias arriba indicados, a fin de tratar el siguiente orden del día:

\tableofcontents

\vspace{1cm}

A sugerencia de la tesorera Sara Báez García, los puntos 1 y 3 se tratarán primero para mayor claridad.

\section{Aprobación, si procede, de las cuentas económicas del ejercicio 2020}

Presentadas y explicadas las cuentas económicas del ejercicio 2020, resumidas en el siguiente cuadro:

\begin{center}
    \begin{tabular}{ | l | r | }
        \hline
        \textbf{Resumen de cuentas económicas} & \textbf{2020} \\ 
        \hline
        Total Ingresos & 1.245,71\euro \\  
        \hline
        Total Gastos & -766,69\euro \\  
        \hline
        \hline
        Diferencia & 479,02\euro \\  
        \hline
    \end{tabular}
\end{center}

Se aprueban las cuentas económicas por unanimidad de la asamblea.

\section{Aprobación, si procede, de la memoria anual de actividades del año 2020}

Dada la escasa actividad durante el año 2020, que ha hecho imposible la realización de ningún evento presencial, no se ha realizado un documento de memorial anual, pero se hace constar en esta reunión que la única actividad realizada durante el año fue un evento de tipo PyBirras en modalidad ``on-line'', realizado a través de la herramienta Zoom.\\

Se aprueba la memoria de actividades por unanimidad de la asamblea.

\section{Aprobación, si procede, del presupuesto para el año 2021}

Presentado y explicado el presupuesto para el año 2021, resumido
en el siguiente cuadro:

\begin{center}
    \begin{tabular}{ | l | r | }
        \hline
        \textbf{Presupuestos} & \textbf{2021} \\ 
        \hline
        Total Ingresos & 235,00\euro \\  
        \hline
        Total Gastos & -470,66\euro \\  
        \hline
        \hline
        Diferencia & 235,66\euro \\  
        \hline
    \end{tabular}
\end{center}

Se aprueba el presupuesto por unanimidad de la asamblea.

\section{Aprobación, si procede, del plan general de actuación para el año 2021}

Dado que la pandemia del COVID-19 aun impide, por el momento, la realización de eventos presenciales, por ahora el único evento planificado será un evento ``on-line'', realizado con un esquema similar, pero con algunas diferencias, a un pyBirras. Las ideas fundamentales de este evento son las siguientes:

\begin{itemize}
    \item Un evento de entre una hora a hora y media de duración.
    \item Un bloque inicial compuesto de dos o tres mini-charlas, cada una de las cuales debería extenderse como máximo 15 minutos, y un tiempo adicional de 5 minutos para preguntas y respuestas.
    \item Las charlas podrian ir englobadas en una temática común. A modo de ejemplo se propuso el tema de ``Cómo gestionar tu flujo de trabajo''. Se comenta la posibilidad de debatir varios temas posibles para usar en esta y en futuras ediciones, si vemos que el formato tiene aceptación.
    \item Por último, una sección final a modo de mesa redonda / tertulia, en la que sería interesante que todos los asistentes pudieran participar, de forma coordinada.
\end{itemize}

Se propone como fecha tentativa para este evento la última semana de junio, quedando a criterio de la Junta Directiva la decisión final sobre su celebración.\\

Se aprueba el plan general de actuación por unanimidad de la asamblea.

\section{Renovación de cargos en la junta directiva}

No se produce ninguna renovación en la junta directiva.

\section{Ratificación de altas y bajas de socios y socias}

Se explica la decisión tomada por la junta directiva de añadir un periodo de gracia de \emph{seis meses} para todos los socias y las socias. Esta decisión se basa tanto en reconocer la poca actividad realizada durante este periodo de pandemia, como en obtener un periodo de tiempo que nos permita mejorar el método de cobro, para simplificar tanto el abono de la anualidad como la incorporación de nuevos socios.

\section{Otra información de interés}

A efectos de prever el impacto económico de la ampliación del periodo de pago de cuota de 6 meses aplicado a todos los socios y socias, se hará un estudio con los datos actualizados de fechas de cobro modificadas, aunque no se espera una gran diferencia con lo presupuestado.

\section{Sugerencias y preguntas}

No hay sugerencias ni preguntas.

% ================================================================================================

\vspace{1cm}
\hrule
\vspace{3mm}

Una vez tratados todos los puntos, se levanta la sesión cuando son las \textit{18:35h} en lugar y fecha arriba indicados.

\vspace{1cm}

\begin{table}[h]
    \begin{tabular}{p{9cm}p{9cm}}
        VºBº Presidencia & El secretario (P.A.) \\
        \vspace{3cm} & \vspace{3cm} \\
        Juan Ignacio Rodríguez de León & Héctor Álvarez Alonso \\
    \end{tabular}
\end{table}

\end{document}
