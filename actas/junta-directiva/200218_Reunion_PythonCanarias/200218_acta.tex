\documentclass[a4paper, 12pt]{article}

\usepackage[T1]{fontenc}
\usepackage[utf8]{inputenc}
\usepackage[spanish]{babel}
\usepackage{float}
\usepackage{hyperref}
\usepackage{tocloft}
\usepackage{titling}
\usepackage{eurosym}
\usepackage{bookmark}

\renewcommand{\labelitemi}{$\bullet$}
\renewcommand{\labelitemii}{$\circ$}
\renewcommand{\labelitemiii}{--}
\renewcommand{\cftsecleader}{\cftdotfill{\cftdotsep}}
\newcommand{\specialcell}[2][c]{%
    \begin{tabular}[#1]{@{}c@{}}#2\end{tabular}
}

\setlength{\droptitle}{-10em}

\hypersetup{
    colorlinks=true,
    linkcolor=black,
    filecolor=magenta,      
    urlcolor=cyan,
}

\hyphenation{Python-Canarias}
\hyphenation{Py-Day}
\hyphenation{Py-thon}

\title{\huge \textbf{Acta de reunión} \\ Junta Directiva \\ \textit{Python Canarias}}
\date{\textbf{18 de febrero de 2020}}
\author{
    Héctor Álvarez Alonso \and
    Juan Ignacio Rodríguez de León \and 
    Sergio Delgado Quintero
}

\begin{document}

\renewcommand{\contentsname}{Orden del día}

\maketitle

Vía \textit{Google Hangouts}, siendo las \textit{19:15h (Atlantic/Canary)} de la fecha arriba indicada se reúnen los miembros de la junta directiva de Python Canarias arriba indicados, a fin de tratar el siguiente orden del día:

\tableofcontents

\section{Lectura y aprobación, si procede, del acta anterior}

Se aprueba el acta anterior.

\section{Cambio en asesoría fiscal}

Tras haber solicitado distintos presupuestos a asesorías fiscales de Tenerife para que nos llevaran la contabilidad oficial y la presentación de impuestos, hemos decidido quedarnos con \href{http://rodriguezbatista.com/}{Rodríguez Batista} principalmente por el coste del servicio, que nos parecía muy razonable.

\subsection{Domicilio social y fiscal}

Actualmente el domicilio social y fiscal de la asociación está establecido en la asesoría Taoro Consultores (Avda. Alonso Fernández de Lugo, 4 - 1C; 38300 La Orotava). Dado que vamos a cambiar de asesoría se hace necesario un cambio de domicilio. En este sentido se baraja la opción de establecerlo en la nueva asesoría \textit{Rodríguez Batista}. Queda pendiente de consulta.

\subsection{Domicilio a efectos de notificaciones}

Se ha hecho la consulta a \textit{Correos España} y sabemos que disponer de un \textit{apartado de correos} cuesta unos 65\euro\ anuales incluyendo servicio de notificaciones vía SMS. Se ve razonable asumir este gasto para poder disponer de un domicilio a efectos de notificaciones.

\section{Difusión del curso Python de la EOI}

Se acuerda aprobar la difusión del \href{http://a.eoi.es/ampz}{curso de especialización en Python} organizado por la \href{https://www.eoi.es/}{Escuela de Organización Industrial} a través de las redes sociales y sistemas de mensajería de nuestra asociación, sin que esta difusión conlleve coste alguno para la propia asociación.

\section{Teléfono de contacto de la asociación}

Tras la propuesta de disponer de un número de teléfono para la asociación, que podría ser incluso dado de alta a través de \textit{tarjeta prepago}, no se ve la necesidad del mismo. Por tanto queda desestimada dicha propuesta ya que la mayoría de contactos que se producen hacia la asociación son a través de correo electrónico, redes sociales o sistemas de mensajería.

\section{Próximos eventos}

Se acuerda organizar un \textit{PyBirras} entre los meses de marzo-abril. Se le quiere dar una vuelta a este evento e ir un poco más allá del formato clásico de charlas. Aún no está claro pero podría enmarcarse en un taller más práctico o incluso una sesión de preguntas que sean respondidas sobre la marcha. Aún quedan pendientes los detalles concretos del evento.

\section{Ruegos y preguntas}

No hay ruegos ni preguntas.

% ================================================================================================

\vspace{1cm}
\hrule
\vspace{3mm}

Una vez tratados todos los puntos, se levanta la sesión cuando son las \textit{20:30h} en lugar y fecha arriba indicados.

\begin{flushright}
El secretario

Sergio Delgado Quintero
\end{flushright}

\end{document}
