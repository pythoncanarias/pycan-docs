\documentclass[a4paper, 12pt]{article}

\usepackage[T1]{fontenc}
\usepackage[utf8]{inputenc}
\usepackage[spanish]{babel}
\usepackage{float}
\usepackage{hyperref}
\usepackage{tocloft}
\usepackage{titling}
\usepackage{eurosym}
\usepackage{bookmark}

\renewcommand{\labelitemi}{$\bullet$}
\renewcommand{\labelitemii}{$\circ$}
\renewcommand{\labelitemiii}{--}
\renewcommand{\cftsecleader}{\cftdotfill{\cftdotsep}}
\newcommand{\specialcell}[2][c]{%
    \begin{tabular}[#1]{@{}c@{}}#2\end{tabular}
}

\setlength{\droptitle}{-10em}

\hypersetup{
    colorlinks=true,
    linkcolor=black,
    filecolor=magenta,      
    urlcolor=cyan,
}

\hyphenation{Python-Canarias}
\hyphenation{Py-Day}
\hyphenation{Py-thon}

\title{\huge \textbf{Acta de reunión} \\ Junta Directiva \\ \textit{Python Canarias}}
\date{\textbf{24 de octubre de 2019}}
\author{
    Juan Ignacio Rodríguez de León \and 
    Sergio Delgado Quintero
}

\begin{document}

\renewcommand{\contentsname}{Orden del día}

\maketitle

Vía \textit{Google Hangouts}, siendo las \textit{20:00h (Atlantic/Canary)} de la fecha arriba indicada se reúnen los miembros de la junta directiva de Python Canarias arriba indicados, a fin de tratar el siguiente orden del día:

\tableofcontents

\section{Lectura y aprobación, si procede, del acta anterior}

Se aprueba el acta anterior.

\section{Formación Python para programa impulsado por la EOI}

\subsection*{Propuesta que nos hacen llegar}

\begin{itemize}
    \item El pasado 14 de octubre se puso en contacto con nosotros \textit{Pablo Velasco Garrido}, Director de Programas de la \href{https://www.eoi.es}{Fundación EOI}, para proponernos lanzar un programa de formación específica en Python.
    \item La EOI tiene como socios en Canarias a SPEGC, INTECH y FIFEDE.
    \item El programa formativo sería enfocado a personas menores de 29 años y desempleadas y está inmerso en el \textit{Sistema Nacional de Garantía Juvenil}. Tendría una duración de entre 200 y 300 horas.
    \item El objetivo de estos programas es reducir el \textit{``gap''} existente entre la formación reglada y la empresa.
    \item Hay que desarrollar una guía de orientación pegadógica (\textit{GOP}) que contendrá la estructura del programa, contenidos, objetivos, secuenciación y temporalización.
    \item Existe la figura de \textit{director del programa académico} que se encargará de desarrollar la GOP, buscar al profesorado para impartir los contenidos y establecer el calendario de aplicación.
    \item Hay un módulo obligatorio en todos los programas que es el de \textit{empleabilidad} (\textit{soft-skills}). Hay una profesora que ya está buscada, quien se encarga de impartir dicho módulo.
    \item Este programa debería contar con una matrícula de entre 12 y 15 asistentes para que salga adelante. Se pueden establecer requisitos de acceso (titulación universitaria, formación profesional, etc.)
    \item La EOI se encarga de buscar el local/aula para impartir el programa.
\end{itemize}

\subsection*{Medidas acordadas}

\begin{itemize}
    \item Crear un listado de personas interesadas en impartir cursos/talleres/programas de formación en Python entre los miembros de la comunidad.
    \item Seleccionar a aquellas personas más adecuadas para impartir este programa y dar respuesta a la EOI en cuanto al calendario de aplicación.
\end{itemize}

\section{Ruegos y preguntas}

No hay ruegos ni preguntas.

% ================================================================================================

\vspace{1cm}
\hrule
\vspace{3mm}

Una vez tratados todos los puntos, se levanta la sesión cuando son las \textit{20:30h} en lugar y fecha arriba indicados.

\begin{flushright}
El secretario

Sergio Delgado Quintero
\end{flushright}

\end{document}
